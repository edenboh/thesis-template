\phantomsection
\section*{Index visuel}
\addcontentsline{toc}{section}{Index visuel}

% === Page 1 ===
\begin{figure}[H]
    \centering
    \includegraphics[width=0.7\textwidth]{images/formulaireDinfo.png}
    \caption{Formulaire de contact}
\label{fig:formulaire-contact}
\end{figure}

\begin{figure}[H]
    \centering
    \includegraphics[width=0.7\textwidth]{images/maisonTemoin.png}
    \caption{Page de maison témoin avec Google Maps}
    \label{fig:maison-temoin}
\end{figure}

\begin{figure}[H]
    \centering
    \includegraphics[width=0.7\textwidth]{images/solar.png}
    \label{fig:analyse-solaire}
    \caption{Page d’analyse solaire}
\end{figure}

\newpage

% === Page 2 ===
\begin{figure}[H]
    \centering
    \includegraphics[width=0.7\textwidth]{images/crmCommerciaux.png}
    \caption{Interface de connexion simplifiée}
    \label{fig:connexion-crm}
\end{figure}

\begin{figure}[H]
    \centering
    \includegraphics[width=0.7\textwidth]{images/pageAcceuilCrm.png}
    \caption{Page d'accueil}
    \label{fig:page-accueil}
\end{figure}

\begin{figure}[H]
    \centering
    \includegraphics[width=0.7\textwidth]{images/filtre2.png}
    \caption{Filtrage des dossiers}
    \label{fig:filtrage}
\end{figure}

\newpage

% === Page 3 ===
\begin{figure}[H]
    \centering
    \includegraphics[width=0.8\textwidth]{images/telepro20.png}
    \caption{Analyse par télépros}

    \label{fig:analyse-telepros}
    \vspace{0.5em}
        {\small
        Ici, on peut voir que sur l’année 2024, Samy Belloc a été le téléprospecteur plus performant.
        Le taux de RDV correspond au nombre de rendez-vous pris divisé par le nombre de leads reçus.
        Il affiche le meilleur taux sur cette période.

        }
\end{figure}


\begin{figure}[H]
    \centering
    \includegraphics[width=0.8\textwidth]{images/commerciaux2.0.png}
    \caption{Analyse par commerciaux}\vspace{0.5em}

    \vspace{0.5em}
            {\small
           Ici, on peut voir que sur l’année 2024, Adrien Grebert a été le commercial le plus performant.
           Le taux de transformation correspond au nombre d’installations divisé par le nombre d’affaires signées.
           Il présente le meilleur taux de l’équipe sur cette période.

            }
\end{figure}


