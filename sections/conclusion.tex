\section{Conclusion}
\label{conclusion}

Cette année d’alternance a renforcé ma conviction quant à l’orientation que je souhaite donner à mon parcours professionnel. En évoluant à la fois en entreprise et à l’université, j’ai pu affiner ma vision du métier de développeuse et mieux comprendre les enjeux du secteur, tant sur le plan technique que stratégique.

J’ai toujours pensé que l’apprentissage était le meilleur moyen de se préparer au monde professionnel, et cette année en est la preuve : elle m’a permis de développer des compétences concrètes, de me confronter à des situations réelles et de gagner en maturité. Ce format m’a également aidée à mieux cerner mes forces, les domaines que je souhaite approfondir, et le type de poste vers lequel je tends.

C’est pour cela que j’ai choisi de poursuivre mes études en \textbf{Master \textsuperscript{22}MIAGE}, une formation qui, selon moi, représente parfaitement l’équilibre entre compétences techniques en informatique et compétences en gestion d’entreprise. Le monde professionnel évolue vers une intégration de plus en plus forte entre ces deux domaines, et je suis convaincue que cette double compétence me donnera les clés pour répondre aux enjeux de demain, notamment dans des postes à responsabilité où la coordination entre technique et stratégie est essentielle.

Mon objectif est de continuer à évoluer dans ce modèle d’alternance, afin de poursuivre cette montée en compétence progressive, et de me préparer à terme à des rôles de cheffe de projet, consultante IT, ou toute autre fonction mêlant pilotage, vision métier et expertise technique.