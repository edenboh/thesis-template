

\section{Contexte d’alternance}
\label{context}

\subsection{Présentation de l’entreprise}

\textbf{Secteur d’activité, taille, organisation}\vspace{0.3cm}

Ecovair est une SAS\textsuperscript{1}, classée PME\textsuperscript{2}, dont le siège social est situé au 37 rue Adam Ledoux, 92026 Courbevoie.

Fondée en 2017 par Benjamin Bohbot, Arié Yahdan et Alexandre Tahar, Ecovair est une entreprise spécialisée dans les énergies renouvelables, notamment dans l'installation de pompes à chaleur (air-eau et air-air) et de panneaux solaires.

L’entreprise propose un accompagnement complet, allant du démarchage commercial à l’installation finale, en passant par l’assistance administrative pour les demandes d’aides gouvernementales. Elle prend également en charge l’entretien des installations, soit via des interventions ponctuelles payantes, soit dans le cadre de contrats d’entretien, qui prévoient des passages réguliers pour la maintenance, la prévention des pannes, ou encore le nettoyage des équipements.

Avec une trentaine d’employés, l’entreprise est structurée en plusieurs pôles aux missions spécifiques :\vspace{0.3cm}

\begin{itemize}
    \item \textbf{Les téléprospecteurs} : ils se chargent de contacter les potentiels clients, de leur présenter les services proposés, ainsi que de poser un rendez-vous avec les commerciaux. Sous la supervision de Benjamin Bohbot.\vspace{0.3cm}
    \item \textbf{Les commerciaux} : ils interviennent après la téléprospection pour se rendre physiquement chez le client, évaluer la faisabilité technique du projet, expliquer en détail l’offre, et faire signer le contrat si tous les critères sont réunis. Sous la supervision de Benjamin Bohbot.\vspace{0.3cm}
    \item \textbf{L’équipe administrative} : elle gère la relation client une fois l’affaire signée et s’occupe des démarches de demande d’aide gouvernementale. Sous la supervision d’Alexandre Tahar.\vspace{0.3cm}
    \item \textbf{Le service après-vente (SAV)} : il traite toutes les réclamations et interventions post-installation. Sous la supervision d’Arié Yahdan.\vspace{0.3cm}
    \item \textbf{Le dépôt et la gestion du stock} : il est responsable de la logistique et de l’approvisionnement en matériel nécessaire. Sous la supervision d’Arié Yahdan.\vspace{0.3cm}
    \item \textbf{Le service informatique} : composé de moi-même et de mon tuteur Avi Mimoun. Sous la supervision de Benjamin Bohbot.\vspace{0.3cm}
\end{itemize} \vspace{0.3cm}

\textbf{Culture d’entreprise}\vspace{0.3cm}

L’entreprise met en place différentes actions en faveur du bien-être de ses salariés, notamment par l’aménagement d’espaces dédiés aux temps de pause, comme une cuisine équipée et des jeux accessibles en libre-service. Ces dispositifs visent à favoriser un environnement de travail détendu et convivial.
\subsection{Présentation de l’équipe}

\textbf{Rôles, interactions et communication}\vspace{0.3cm}

Les relations au sein de l’entreprise sont très bonnes. Il y a une excellente communication, du fait que les trois pôles principaux sont présents dans le même bureau.

De plus, étant une petite structure, une trentaine de collaborateurs, cela permet des relations harmonieuses et une collaboration efficace entre les équipes.

Les téléprospecteurs initient le contact client, l’équipe administrative prend le relais pour la gestion des dossiers, le SAV assure le suivi post-installation, tandis que le service informatique, auquel je contribue, soutient la digitalisation des processus.

L’utilisation d’un outil \textsuperscript{3}CRM (\textit{Zoho}) renforce cette coordination en centralisant les données clients et en facilitant le suivi des affaires, améliorant ainsi les interactions entre les pôles.

\subsection{Les besoins de l’entreprise}

\textbf{Besoins généraux et informatiques}\vspace{0.3cm}

Dans un secteur parfois associé à des pratiques critiquables, Ecovair cherche à se distinguer par une approche plus rigoureuse et transparente. Il est essentiel de rassurer les clients potentiels et de gagner leur confiance, en mettant en place des démarches claires et cohérentes à tous les niveaux de l’entreprise.

Pour cela, l’entreprise exprime un besoin fort de structuration et de professionnalisation, notamment à travers :\vspace{0.3cm}

\begin{itemize}
    \item L’amélioration continue de la qualité de service et du suivi client.\vspace{0.3cm}
    \item La mise en œuvre de processus internes efficaces et transparents, afin de renforcer la crédibilité de l’entreprise.\vspace{0.3cm}
    \item Une communication claire, rapide et cohérente entre les différents pôles, pour éviter les erreurs et offrir une expérience fluide aux clients.\vspace{0.3cm}
    \item Une présence numérique maîtrisée et professionnelle, qui reflète le sérieux et l’engagement de l’entreprise.\vspace{0.3cm}
\end{itemize}

C’est dans cette optique qu’Ecovair mise sur une digitalisation poussée de ses activités. L’objectif est d’optimiser l’organisation interne, d’automatiser les tâches répétitives, de centraliser les données, et de renforcer l’image de modernité et de fiabilité de l’entreprise auprès de ses clients et partenaires.\vspace{0.3cm}

\textbf{Problématiques et attentes vis-à-vis de mon poste}\vspace{0.3cm}

En tant qu'alternante au sein de l'entreprise, je contribue donc au développement et à l'optimisation des outils numériques, notamment en créant et en maintenant le site internet \texttt{ecovair.fr}, tout en intégrant des solutions technologiques avancées.

Je participe à la simplification des processus pour les équipes commerciales, à l'amélioration des interactions avec les clients grâce à des solutions d'intelligence artificielle, et à la création de rapports de \textit{Business Intelligence} pour appuyer les décisions stratégiques.



\section{Mes missions}
\label{missions}

\subsection{Organisation de travail}

Dans le cadre de mon alternance, une organisation rigoureuse est essentielle pour gérer efficacement la diversité des projets qui me sont confiés. Mes tâches, souvent variées et évolutives, exigent une grande capacité d’adaptation et une gestion optimale de mon temps, car je travaille rarement sur un seul projet à la fois.

Chaque nouveau projet débute par une réunion organisée par mon tuteur ou notre responsable, en présentiel ou à distance selon les disponibilités. Cette réunion vise à définir les objectifs, clarifier les attentes et discuter des technologies et langages à utiliser, en échangeant sur les solutions techniques les plus adaptées.

Une fois le projet lancé, je dispose d’une vision claire de mes responsabilités et travaille de manière autonome, tout en pouvant compter sur le soutien de mon tuteur pour des conseils ou des retours constructifs. Des points réguliers sont organisés pour suivre l’avancement de mes travaux. Lorsqu’un risque de sécurité est identifié, je soumets mon code à mon tuteur pour relecture, ce qui me permet de corriger d’éventuelles failles et d’améliorer la robustesse des applications développées.

Sans deadlines strictes, je gère mon temps en fixant mes propres objectifs d’avancement. Pour cela, j’utilise \textit{\textsuperscript{4}Notion} comme outil d’organisation, structurant mes tâches en trois catégories~: \textit{À commencer}, \textit{En cours} et \textit{Terminées}. Cette méthode me permet de visualiser l’état d’avancement de mes projets, de conserver un historique clair de mes réalisations et de préparer efficacement les points de suivi avec mon tuteur.

Au quotidien, je planifie mes journées en définissant des objectifs précis pour la matinée et l’après-midi. Cette organisation favorise ma concentration, équilibre ma charge de travail et maintient une dynamique productive.

\subsection{Langages et environnements utilisés}

Les langages et \textsuperscript{5}frameworks mobilisés dans mes missions incluent~:\vspace{0.3cm}

\begin{itemize}
    \item \textbf{React} et \textbf{Next.js} (\textsuperscript{5}frameworks JavaScript pour le développement d’interfaces dynamiques).\vspace{0.3cm}
    \item \textbf{JavaScript} (Node.js pour le backend).\vspace{0.3cm}
    \item \textbf{PHP} (Symfony pour le développement d’applications web).\vspace{0.3cm}
    \item \textbf{\textsuperscript{6}COQL} et \textbf{\textsuperscript{7}SQL} (pour la gestion des bases de données).\vspace{0.3cm}
    \item \textbf{HTML} et \textbf{CSS} (avec \textsuperscript{8}Bootstrap pour le design responsive).\vspace{0.3cm}
\end{itemize}

Les environnements de développement utilisés sont~:\vspace{0.3cm}

\begin{itemize}
    \item \textbf{\textsuperscript{9}JetBrains IntelliJ} et \textbf{PhpStorm} (pour le développement).\vspace{0.3cm}
    \item \textbf{Zoho \textsuperscript{3}CRM} (pour la gestion des données clients).\vspace{0.3cm}
    \item \textbf{Vercel} et \textbf{Google Cloud Run} (pour le déploiement des applications).\vspace{0.3cm}
\end{itemize}

\subsection{Git}

Git, développé en 2005 par Linus Torvalds, est le système de contrôle de version le plus utilisé au monde. Ce projet open source est apprécié pour sa fiabilité et ses performances. Ses principales fonctionnalités incluent~:\vspace{0.3cm}

\begin{itemize}
    \item La gestion des branches (\texttt{git branch}) pour travailler sur plusieurs versions d’un projet en parallèle.\vspace{0.3cm}
    \item La création d’instantanés du code (\texttt{git commit}).\vspace{0.3cm}
    \item La fusion des branches (\texttt{git merge}).\vspace{0.3cm}
    \item La publication rapide des modifications sur un dépôt distant (\texttt{git push}).\vspace{0.3cm}
\end{itemize}

Chez Ecovair, tous les projets sont centralisés sur \textit{GitHub}, garantissant une organisation efficace, une traçabilité des modifications et une continuité de service.

\subsection{Mes missions}

\subsubsection{Création et maintenance du site principal d’Ecovair (\texttt{ecovair.fr})}

\textbf{Contexte de la mission}\vspace{0.3cm}

Depuis le début de mon alternance, je contribue à la maintenance et à l’évolution du site principal d’Ecovair, \texttt{ecovair.fr}. L’objectif est de moderniser son apparence, de renforcer la crédibilité de l’entreprise et d’optimiser l’expérience utilisateur pour faciliter les interactions avec les clients potentiels.\vspace{0.3cm}

\textbf{Objectifs de la mission}\vspace{0.3cm}

\begin{itemize}
    \item Moderniser le design pour une interface plus professionnelle, fluide et accessible.\vspace{0.3cm}
    \item Intégrer des fonctionnalités interactives connectées à des services externes (\textsuperscript{10}API) et au \textsuperscript{3}CRM.\vspace{0.3cm}
    \item Renforcer l’image de fiabilité et de professionnalisme auprès des clients.\vspace{0.3cm}
\end{itemize}

\textbf{Réalisations de la première année (N-1)}\vspace{0.3cm}

\begin{enumerate}
    \item \textbf{Refonte du design} : Modernisation de l’apparence du site pour le rendre plus lisible, fluide et accessible.\vspace{0.3cm}
    \item \textbf{Formulaire de contact} : Mise en place d’un formulaire intuitif permettant aux visiteurs de transmettre leurs coordonnées, facilitant la prise en charge par l’équipe commerciale.\vspace{0.3cm}
\end{enumerate}

\textbf{Réalisations de la deuxième année}\vspace{0.3cm}

\begin{enumerate}
    \item \textbf{Page de maison témoin avec Google Maps} : Développement d’une page permettant aux utilisateurs de saisir leur code postal pour visualiser, via l’\textsuperscript{10}API Google Maps, les installations réalisées dans leur secteur. Les données sont extraites dynamiquement du \textsuperscript{3}CRM.\vspace{0.3cm}
    \item \textbf{Page d’analyse solaire} : Création d’une page utilisant l’\textsuperscript{10}API Solar de Google pour évaluer l’ensoleillement d’un bâtiment et vérifier la faisabilité d’une installation de panneaux solaires.\vspace{0.3cm}
\end{enumerate}

Ces améliorations renforcent l’interactivité du site et l’image professionnelle d’Ecovair tout en offrant une expérience utilisateur fluide.

\subsubsection{Développement d’un portail de suivi des dossiers pour les commerciaux}

\textbf{Contexte de la mission}\vspace{0.3cm}

La gestion du \textsuperscript{3}CRM Zoho pose deux défis~: un coût élevé par utilisateur (20 à 60 euros par mois) et une interface trop complexe pour les besoins des commerciaux, qui se concentrent principalement sur le suivi des dossiers clients. Offrir un accès complet au \textsuperscript{3}CRM à toute l’équipe commerciale est donc coûteux et peu pertinent.

L’objectif était de développer une solution alternative permettant aux commerciaux de suivre efficacement leurs dossiers sans accès direct au \textsuperscript{3}CRM, tout en respectant les contraintes budgétaires.\vspace{0.3cm}

\textbf{Objectifs de la mission}\vspace{0.3cm}

\begin{itemize}
    \item Créer une interface dédiée, simple et sécurisée, pour le suivi des dossiers.\vspace{0.3cm}
    \item Limiter l’accès aux seules données nécessaires, réduisant ainsi les coûts et les risques.\vspace{0.3cm}
    \item Optimiser la gestion des dossiers et des primes des commerciaux.\vspace{0.3cm}
\end{itemize}

\textbf{Réalisations}\vspace{0.3cm}

J’ai développé un portail web avec le framework \textit{Symfony} (PHP), intégrant les fonctionnalités suivantes~:\vspace{0.3cm}

\begin{enumerate}
    \item \textbf{Interface de connexion simplifiée} : Authentification via un lien temporaire sécurisé (\textit{\textsuperscript{11}magic link}) envoyé par e-mail, éliminant la gestion de mots de passe.\vspace{0.3cm}
    \item \textbf{Filtrage des dossiers} : Fonctionnalité permettant aux commerciaux de filtrer leurs dossiers selon des critères comme le statut ou le type de service, avec des données extraites via l’\textsuperscript{10}API REST de Zoho \textsuperscript{3}CRM.\vspace{0.3cm}
    \item \textbf{Accès limité aux données} : Affichage exclusif des informations pertinentes, garantissant sécurité et simplicité.\vspace{0.3cm}
\end{enumerate}

Cette solution permet aux commerciaux de suivre efficacement leurs dossiers, améliore la gestion de leurs primes et respecte les contraintes budgétaires de l’entreprise.


\subsubsection{Exploitation de Zoho \textsuperscript{12}Analytics pour la Business Intelligence commerciale}

\textbf{Contexte de la mission}\vspace{0.3cm}

Ecovair stocke une grande quantité de données relatives aux dossiers clients dans Zoho \textsuperscript{3}CRM, incluant l’identité des téléprospecteurs et commerciaux, le nombre de dossiers annulés ou non signés, et d’autres informations clés. Ces données sont essentielles, car la rémunération des téléprospecteurs et commerciaux dépend de leurs performances. Raphaël Tobelem, responsable des téléprospecteurs, a besoin d’outils pour suivre les performances de son équipe, identifier les collaborateurs les plus performants et ajuster les stratégies commerciales.

L’enjeu est de disposer d’une solution simple et efficace pour analyser ces données régulièrement, avec des filtres par période pour une analyse fine, permettant d’identifier des corrélations, comme l’impact des jours de rendez-vous sur le taux de signature des dossiers.\vspace{0.3cm}

\textbf{Objectifs de la mission}\vspace{0.3cm}

\begin{itemize}
    \item Créer des tableaux de bord interactifs pour analyser les performances des équipes.\vspace{0.3cm}
    \item Faciliter le suivi des performances individuelles pour optimiser la gestion des primes.\vspace{0.3cm}
    \item Permettre une analyse détaillée des données sur différentes périodes.\vspace{0.3cm}
\end{itemize}

\textbf{Réalisations}\vspace{0.3cm}

J’ai utilisé \textit{Zoho  \textsuperscript{12}Analytics}, l’outil de Business Intelligence intégré à Zoho, pour développer des solutions adaptées~:

\begin{enumerate}
    \item \textbf{Requêtes \textsuperscript{7}SQL} : Rédaction de requêtes \textsuperscript{7}SQL pour extraire et filtrer précisément les données de Zoho \textsuperscript{3}CRM, permettant une analyse par période (mensuelle, trimestrielle, etc.).\vspace{0.3cm}
    \item \textbf{Dashboards interactifs} : Conception de tableaux de bord dans  Zoho \textsuperscript{12}Analytics, affichant des indicateurs clés comme les performances des téléprospecteurs et commerciaux, le nombre de dossiers annulés ou non signés. Ces dashboards offrent une visualisation claire et synthétique.\vspace{0.3cm}
    \item \textbf{Suivi des performances individuelles} : Mise en place de fonctionnalités permettant de suivre les performances individuelles, facilitant les décisions sur les primes et l’accompagnement des équipes.\vspace{0.3cm}
\end{enumerate}

Cette solution améliore le suivi des performances commerciales, optimise les processus décisionnels et permet une meilleure adaptation des stratégies grâce aux données analysées.

\subsubsection{Mise en place d’une solution d’échange avec les clients basée sur l’intelligence artificielle}

\textbf{Contexte de la mission}\vspace{0.3cm}

Dans le cadre de notre veille technologique, nous avons exploré l’\textsuperscript{10}API \textit{real-time} d’\textsuperscript{13}OpenAI, permettant des interactions vocales en temps réel (\textit{speech-to-speech}), en partenariat avec \textsuperscript{14}Twilio, une plateforme cloud de communication. L’objectif était d’exploiter cette technologie pour améliorer la relation client, notamment en automatisant les interactions avec des populations moins réceptives aux outils numériques, comme les personnes âgées.\vspace{0.3cm}

\textbf{Objectifs de la mission}\vspace{0.3cm}

\begin{itemize}
    \item Automatiser les enquêtes de satisfaction par des appels vocaux gérés par une intelligence artificielle.\vspace{0.3cm}
    \item Recueillir des retours clients exploitables pour enrichir le \textsuperscript{3}CRM.\vspace{0.3cm}
    \item Explorer des cas d’usage pour optimiser la relation client et poser les bases d’une assistance technique automatisée.\vspace{0.3cm}
\end{itemize}

\textbf{Réflexion stratégique sur les cas d’usage}\vspace{0.3cm}

Avant de prioriser les enquêtes de satisfaction, nous avons envisagé d’utiliser l’intelligence artificielle comme assistant technique, permettant aux clients de poser des questions sur les équipements via un numéro dédié. Cependant, ce cas d’usage nécessitait un travail préalable de structuration des données (fiches techniques, notices, etc.), trop chronophage à ce stade. Nous avons donc opté pour les enquêtes de satisfaction, plus simples à mettre en œuvre, comme première application.\vspace{0.3cm}

\textbf{Problèmes rencontrés et solutions apportées}\vspace{0.3cm}

Un défi majeur était l’incompatibilité de l’option d’enregistrement native de \textsuperscript{14}Twilio avec l’\textsuperscript{10}API \textit{Speech-to-Speech} d’\textsuperscript{13}OpenAI, qui utilise \textsuperscript{16}WebSocket pour le traitement en temps réel. Cela empêchait l’enregistrement des conversations, essentiel pour analyser les retours clients. De plus, les premières versions des enregistrements souffraient de problèmes de qualité sonore (illisibilité, accélération ou brouillage).\vspace{0.3cm}

Pour résoudre ces problèmes~:

\begin{enumerate}
    \item \textbf{Enregistrement personnalisé} : Mise en place d’un système d’enregistrement utilisant les flux audio en direct (\textit{media-stream}) via \textsuperscript{16}WebSocket. Les flux, encodés en G.711 \textmu-\textsuperscript{17}law, ont été convertis en \textsuperscript{18}PCM linéaire 16 bits avec la bibliothèque \texttt{g711}, puis stockés en buffers et enregistrés en fichiers \textsuperscript{19}WAV via la bibliothèque \texttt{wav}.\vspace{0.3cm}
    \item \textbf{Transcription et analyse} : Intégration de l’\textsuperscript{10}API \textit{Whisper} d’\textsuperscript{13}OpenAI pour retranscrire automatiquement les réponses orales en texte, et de l’\textsuperscript{10}API \textit{Completion} pour résumer les échanges, identifiant les retours positifs ou négatifs.\vspace{0.3cm}
\end{enumerate}

\textbf{Impact sur l’entreprise}\vspace{0.3cm}

Bien qu’en phase de finalisation, ce projet offre des perspectives prometteuses~:

\begin{itemize}
    \item \textbf{Accessibilité accrue} : Les enquêtes vocales automatisées touchent un public, notamment âgé, peu réceptif aux e-mails, augmentant le volume de retours collectés.\vspace{0.3cm}
    \item \textbf{Efficacité opérationnelle} : L’automatisation réduit la charge de travail des équipes, libérant des ressources humaines.\vspace{0.3cm}
    \item \textbf{Enrichissement du \textsuperscript{3}CRM} : Les retours collectés pourront être intégrés au \textsuperscript{3}CRM pour enrichir les fiches clients et déclencher des actions ciblées (ex.~: demandes d’avis sur Google ou \textsuperscript{20}Trustpilot).\vspace{0.3cm}
    \item \textbf{Perspectives futures} : Ce projet pose les bases d’une assistance technique automatisée, envisageable à moyen terme.\vspace{0.3cm}
\end{itemize}

Les limites actuelles incluent le besoin d’optimiser la qualité sonore et de finaliser l’intégration des données dans le \textsuperscript{3}CRM, mais les premiers résultats confirment le potentiel de cette solution pour transformer la relation client.






\section{Bilan de l’année}
\label{bilan}

\subsection{Compétences techniques}

Cette année d’alternance m’a permis de consolider et d’élargir mes compétences techniques, tout en gagnant en autonomie et en responsabilité dans la gestion de projets. Bien que j’aie déjà effectué mon alternance dans la même entreprise l’année précédente, j’ai réellement senti une évolution dans la confiance qu’on m’a accordée ainsi que dans la nature des missions confiées.

Sur le plan technique, j’ai approfondi mes connaissances dans plusieurs technologies et \textsuperscript{5}frameworks, notamment \textbf{React}, \textbf{Next.js} et \textbf{Symfony}, que j’ai appris à utiliser de manière plus concrète et efficace dans un contexte professionnel. Ces apprentissages m’ont permis de contribuer à des projets réels, souvent en autonomie, tout en ayant la possibilité de solliciter mon tuteur en cas de besoin. Ce cadre m’a offert un équilibre entre autonomie et accompagnement, favorisant ainsi un apprentissage pratique et progressif.

J’ai également pu mettre en application des compétences acquises à l’université, comme la manipulation de bases de données relationnelles. Un exemple significatif de cette mise en pratique est une mission de \textit{Business Intelligence} menée avec \textbf{ Zoho \textsuperscript{12}Analytics} (décrite précédemment). Ce projet m’a permis de manipuler de vraies données d’entreprise, de comprendre les enjeux liés à leur exploitation, et de produire un outil utile pour le pilotage stratégique.

Par ailleurs, certaines notions plus théoriques, notamment abordées dans les cours de \textit{sociologie des organisations}, ont trouvé une résonance dans mon expérience professionnelle. J’ai compris l’importance d’un environnement de travail sain et stable, propice à l’épanouissement des collaborateurs, et j’ai observé comment cette dimension influait concrètement sur la motivation et la productivité des équipes.

Enfin, au-delà des aspects purement techniques, cette année m’a permis de développer des compétences transversales telles que la \textbf{gestion de projet}, l’\textbf{autonomie}, la \textbf{rigueur}, et la capacité à structurer mon travail de manière efficace.

\subsection{Compétences relationnelles}

Au-delà des compétences techniques, cette année d’alternance m’a également permis de progresser sur le plan relationnel. Naturellement discrète et plutôt timide, j’ai su progressivement m’ouvrir davantage aux autres et trouver ma place au sein de l’équipe. Le fait d’évoluer dans un environnement de travail bienveillant, où je me sens écoutée et utile, a largement contribué à mon épanouissement personnel et professionnel.

J’ai appris à mieux communiquer, à formuler mes idées et mes besoins, notamment lors des réunions de lancement de projet ou des points de suivi avec mon tuteur. Ces échanges réguliers m’ont aidée à gagner en assurance, à m’exprimer plus facilement et à assumer davantage mes responsabilités.

J’ai également compris l’importance de la collaboration, même dans des projets menés en autonomie. Savoir demander de l’aide, recueillir des retours constructifs ou simplement partager l’avancement de mon travail sont devenus pour moi des réflexes naturels. Cela m’a permis non seulement d’améliorer la qualité de mon travail, mais aussi de renforcer mes liens professionnels au sein de l’équipe.

Cette évolution relationnelle représente, pour moi, un véritable progrès : elle m’aide à mieux m’intégrer, à collaborer de manière plus fluide et à me sentir pleinement investie dans mon rôle.

\subsection{Difficultés rencontrées et apprentissages}

Cette année a été particulièrement intense, tant sur le plan académique que professionnel, et m’a confrontée à plusieurs défis qui m’ont permis de grandir et de renforcer ma motivation.

Le passage du \textsc{\textsuperscript{21}BTS} à l’université, et notamment à l’Université Paris Dauphine, a représenté un véritable changement de rythme et d’exigence. L’un des plus grands défis pour moi a été de rattraper mon retard dans les cours de mathématiques, ce qui m’a demandé un effort important. J’ai dû redoubler de travail et de rigueur pour atteindre le niveau attendu, mais je n’ai jamais baissé les bras. Au contraire, j’ai choisi d’investir davantage de temps dans ces matières pour progresser, et je suis fière d’avoir su faire face à cette difficulté avec détermination.

En parallèle, j’ai trouvé beaucoup plus d’intérêt dans les matières orientées vers l’informatique, comme les bases de données relationnelles, les algorithmes dans les graphes, l'Ingénierie des Systèmes d’Information ou encore le Java. Ces enseignements ont nettement enrichi mes compétences par rapport à mon année de \textsuperscript{21}BTS, et m’ont confortée dans mon choix de poursuivre mes études à l’université. Je ne regrette absolument pas ce choix : j’ai réellement senti un saut en termes de qualité d’enseignement, notamment grâce à des professeurs passionnés et investis.

J’ai également beaucoup apprécié des matières moins techniques mais tout aussi enrichissantes, comme \textit{Critical Thinking} ou \textit{Communication}. Ce sont, selon moi, des enseignements qui apportent une réelle valeur dans notre future carrière professionnelle, mais aussi dans notre vie personnelle. Ce sont des acquis précieux que je compte conserver et continuer à développer.

Concernant l’alternance, bien que ce soit ma troisième année dans ce format, j’ai trouvé le rythme cette année plus difficile à tenir. Cela s’explique sans doute par un sentiment de retard sur certaines matières universitaires, combiné à une charge de travail plus importante en entreprise. On m’a confié davantage de responsabilités, ce qui m’a demandé une meilleure organisation et plus d’implication. Je suis néanmoins très reconnaissante de cette confiance, car elle m’a permis d’apprendre énormément et de me confronter à des situations professionnelles plus complexes et formatrices.

En résumé, même si cette année a été exigeante, elle m’a beaucoup appris. Ces difficultés, loin d’être des obstacles, ont été de véritables leviers de progression, tant sur le plan technique que personnel.

