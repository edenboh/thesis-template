\section*{Définitions}

\begin{small}
    \noindent \vspace{0.3cm}
    \textbf{\textsuperscript{1}SAS}: Société par Actions Simplifiée, une forme juridique d'entreprise en France, flexible pour les actionnaires \\\vspace{0.3cm}

    \textbf{\textsuperscript{2}PME}: Petite et Moyenne Entreprise, entreprise de taille modeste avec un effectif et un chiffre d'affaires limités. \\\vspace{0.3cm}

    \textbf{\textsuperscript{3}CRM}: Customer Relationship Management, outil pour gérer les relations et les données des clients.\\\vspace{0.3cm}

    \textbf{\textsuperscript{4}Notion}: Logiciel de productivité pour organiser tâches, notes et projets collaboratifs. \\\vspace{0.3cm}

    \textbf{\textsuperscript{5}Framework}: Cadre logiciel fournissant une structure pour développer des applications. \\\vspace{0.3cm}

     \textbf{\textsuperscript{6}COQL} Custom Object Query Language, langage pour interroger des objets personnalisés, souvent dans des CRM.\\\vspace{0.3cm}

     \textbf{\textsuperscript{7}SQL}: Structured Query Language, langage pour gérer et interroger des bases de données relationnelles. \\\vspace{0.3cm}

    \textbf{\textsuperscript{8}Bootstrap}: Framework CSS et JavaScript pour créer des interfaces web responsives. \\\vspace{0.3cm}

    \textbf{\textsuperscript{9}JetBrains}: Entreprise développant des IDE(Integrated Development Environment) comme IntelliJ IDEA pour programmer\\\vspace{0.3cm}

    \textbf{\textsuperscript{10}API}: Application Programming Interface, interface permettant à des applications de communiquer entre elles.\\\vspace{0.3cm}

    \textbf{\textsuperscript{11}magic link} Lien sécurisé envoyé par e-mail pour se connecter sans mot de passe.\\\vspace{0.3cm}

   \textbf{\textsuperscript{12}Zoho Analytics }: Plateforme d'analyse de données pour créer des rapports et visualisations.\\\vspace{0.3cm}

      \textbf{\textsuperscript{13}OpenAI }:  Organisation développant des technologies d'intelligence artificielle.\\\vspace{0.3cm}

      \textbf{\textsuperscript{14}Twilio }: Plateforme pour intégrer des services de communication (SMS, appels) dans des applications.\\\vspace{0.3cm}

     \textbf{\textsuperscript{16}WebSocket}: Protocole pour une communication bidirectionnelle en temps réel sur le web.\\\vspace{0.3cm}

     \textbf{\textsuperscript{17}G.711 \textmu-law}:Standard de compression audio utilisé en téléphonie pour coder le son (lié à PCM).\\\vspace{0.3cm}

     \textbf{\textsuperscript{18}PCM}: Pulse Code Modulation, méthode de codage numérique pour représenter des signaux audio.\\\vspace{0.3cm}

    \textbf{\textsuperscript{19}WAV} : Format de fichier audio stockant des données non compressées, souvent basé sur PCM.\\\vspace{0.3cm}

    \textbf{\textsuperscript{20}Trustpilot}: Plateforme de collecte d'avis clients pour évaluer la réputation des entreprises.\\\vspace{0.3cm}

     \textbf{\textsuperscript{21}BTS}: Brevet de Technicien Supérieur, diplôme professionnel de niveau post-bac en France. \\\vspace{0.3cm}

     \textbf{\textsuperscript{22}MIAGE} Méthodes Informatiques Appliquées à la Gestion des Entreprises. \\\vspace{0.3cm}


\end{small}
